\documentclass[runningheads]{llncs}
%% The amssymb package provides various useful mathematical symbols
\usepackage{amssymb}
\usepackage{amsmath}
%% The amsthm package provides extended theorem environments
%\usepackage{amsthm}
\usepackage{wrapfig}
\usepackage{lineno}

%% Use package enumitem to align enumeration and itemization
\usepackage{enumitem}
\usepackage{listings}
\usepackage{courier}           % for the courier font (optional)
\usepackage{multicol}          % for two equations side by side
\usepackage[justification=centering]{caption}
\usepackage{xcolor}
\usepackage{stmaryrd}
\usepackage{graphicx}
%\usepackage{url}
\usepackage[misc,geometry]{ifsym} 
\chardef\_=`_

%\hypersetup{ % play with these to change the look of hyperlinks
%    colorlinks=true,
%    linkcolor=black,
%    filecolor=magenta,
%    urlcolor=cyan,
%   citecolor=black
%}

\colorlet{red}{red!80!black}

%% NEW COMMANDS =============================================

\lstdefinelanguage{Coq}%
  {morekeywords={Notation,Class,Variable,Inductive,CoInductive,Fixpoint,CoFixpoint,%
      Definition,Example,Lemma,Theorem,Axiom,Local,Save,Grammar,Syntax,Intro,%
      Trivial,Qed,Intros,Symmetry,Simpl,Rewrite,Apply,Elim,Assumption,%
      Left,Cut,Case,Auto,Unfold,Exact,Right,Hypothesis,Pattern,Destruct,%
      Constructor,Defined,Fix,Record,Print,Proof,Induction,Hint,Hints,Exists,
      let,in,Parameter,Split,Red,Reflexivity,Transitivity,if,then,else,Opaque,%
      Transparent,Inversion,Absurd,Generalize,Mutual,Cases,of,end,Analyze,%
      AutoRewrite,Functional,Scheme,params,Refine,using,Discriminate,Try,%
      Require,Load,Import,Scope,Set,Open,Section,End,match,with,Ltac,%
      exists,forall,fix,fun
	},% 
   keywordstyle=\color{red},%
   basicstyle=\normalfont\footnotesize\tt,%
   sensitive, %
   tabsize=2,%
   escapechar=`,%
   morecomment=[n]{(*}{*)},%
   morestring=[d]",%
   literate={=>}{{$\Rightarrow$}}1 {>->}{{$\rightarrowtail$}}2{->}{{$\to$}}1
   {\/\\}{{$\wedge$}}1
   {|-}{{$\vdash$}}1
   {\\\/}{{$\vee$}}1
   {~}{{$\sim$}}1
   %{<>}{{$\neq$}}1 indeed... no.
  }[keywords,comments,strings]%

\lstnewenvironment{coq}{
       \lstset{
        basicstyle=\ttfamily,
        language=Coq,
        showstringspaces=false
       }
}{}
\makeatletter
\newlength{\@mli}
\newcommand{\mli}[1]{%
	\settowidth{\@mli}{\lstinline/#1/}
	\hspace{-.5ex}\begin{minipage}[t]{\@mli}\lstinline/#1/\end{minipage}}
\makeatother
\newcommand{\li}[1]{\ifmmode\mbox{\mli{#1}}\else\mbox{\lstinline/#1/}\fi}

\newcommand\hide[1]{}
\newenvironment{centermath}
{\begin{center}$\displaystyle}
	{$\end{center}}
\renewcommand{\note}[2][polish]{{\color{red} #2}{\marginpar{\tiny \color{blue} #1}}}
\renewcommand{\implies}{\Rightarrow}
\renewcommand{\iff}{\Leftrightarrow}

%\theoremstyle{plain}
%\newtheorem{thm}{Theorem}[section]
%\newtheorem{ex}[thm]{Example}
%\newtheorem{prop}[thm]{Proposition}
%\newtheorem{col}[thm]{Corollary}
%\newtheorem{lem}[thm]{Lemma}
%\newtheorem*{hypo}{Hypothesis}
%\theoremstyle{definition}
%\newtheorem{defn}[thm]{Definition}
%\newtheorem{rem}[thm]{Remark}

%\crefname{thm}{theorem}{theorems}
%\crefname{defn}{definition}{definitions}
%\crefname{lem}{lemma}{lemmas}
%\crefname{col}{corollary}{corollaries}
%\crefname{ex}{example}{examples}
%\crefname{prop}{proposition}{propositions}

%\Crefname{thm}{Theorem}{Theorems}
%\Crefname{defn}{Definition}{Definitions}
%\Crefname{lem}{Lemma}{Lemmas}
%\Crefname{col}{Corollary}{Corollaries}
%\Crefname{ex}{Example}{Examples}
%\Crefname{prop}{Proposition}{Propositions}

%% DOCUMENT =================================================

\bibliographystyle{plainurl}% the mandatory bibstyle

\begin{document}
	
\title{Formalizing Correct-By-Construction Casper in Coq}

\institute{Runtime Verification}

\maketitle

\section{Overview}

We present a machine-checked formalization of the Correct-By-Construction Casper (CBC Casper) family of consensus protocols in the Coq proof assistant. We leverage Coq's type classes to represent the family of CBC Casper protocols as a collection of parameters and axioms, from which we derive safety and non-triviality properties that hold for each instantiation of the type class using concrete data types and proofs. In doing so, we 1) illuminate the assumptions required of each desired property, and 2) reformulate the protocol in more general mathematical terms. We highlight two advantages of our approach: from a proof engineering perspective, it gives rise to a separation of concerns between theory and implementation; from a protocol engineering perspective, it provides us with a rigorous, foundational understanding of the protocol which allows us to derive stronger properties. 

\section{Formalization} 
In this section we present a Coq type class \verb|CBC_protocol_eq| which defines an abstract CBC Casper protocol. We explicate the type class in two parts: 1) the data types and assumptions it contains corresponding to the parameters identifying each CBC Casper protocol, and 2) the abstract safety proof it derives which is satisfied by each CBC Casper protocol. 
\subsection{Parameters} 
Each member of the family of CBC Casper protocols is identified by five parameters, namely validator names, validator weights, fault tolerance threshold, consensus values, and an estimator function \cite{CBCfull}. Implicitly, each parameter satisfies certain properties, such as non-negativity for validator weights and the fault tolerance threshold, totality for the estimator function etc. We model these parameters as generic \verb|Type|s in Coq, accompanied by explicit statements of their additional properties. 

\subsubsection{Validators} 
Validators represent the consensus-forming nodes in the protocol. We require of validators the fact that 1) they form a non-empty set, namely that there is at least one validator, and 2) they are comparable. We capture both of these requirements in a generic type class \verb|StrictlyComparable|, which is parameterized by an abstract type \verb|X|. 
\begin{coq}
	Class StrictlyComparable (X : Type) : Type :=
	{
	inhabited : exists (x : X), True;
	compare : X -> X -> comparison;
	compare_strictorder :> CompareStrictOrder compare;
	}.
\end{coq}
\verb|StrictlyComparable| contains a proof that there exists at least one witness of type \verb|X| (\verb|inhabited|), a comparison operator which takes two elements of type \verb|X| and returns either \verb|Lt|, \verb|Eq| or \verb|Gt| (\verb|compare|), and a proof that this comparison operator imposes a strict ordering on elements of type \verb|X|, i.e. that it is both reflexive and transitive (\verb|compare_strictorder|). 

This is reflected in the type class as follows: 
\begin{coq}
	validators : Type; 
	about_validators : StrictlyComparable validators;
\end{coq}

\subsubsection{Consensus values} 
The requirements for consensus values represented as abstract types are similar. 
\begin{coq}
	consensus_values : Type; 
	about_consensus_values : StrictlyComparable consensus_values; 
\end{coq}

\subsubsection{Validator weights} 
Validator weights are computed by a function which takes a validator name and returns a positive real number. We represent positive real numbers using dependent types in Coq, and represent the validator weight function with the following function signature: 
\begin{coq}
	weight : validators -> {r | (r > 0)%R};
\end{coq}

\subsubsection{Fault tolerance threshold} 
The Byzantine-fault tolerance threshold is a non-negative real number that is strictly smaller than the sum of all validator weights. Because Coq \verb|Type|s are not guaranteed finite, we state the strictly smaller requirement in terms of the existence of a finite list of validators whose sum exceeds the fault weight. 
\begin{coq}
	t : {r | (r >= 0)%R}; 
	suff_val : exists vs, 
					NoDup vs /\ 
					((fold_right (fun v r => (proj1_sig (weight v) + r)%R) 0%R) vs 
				> (proj1_sig t))%R;
\end{coq}

\subsubsection{States} 
In preparation for defining the estimator function and protocol states, we define a preliminary abstract state type, 
\verb|state|. We similarly require that these states are \verb|StrictlyComparable|. We define an initial state, \verb|state0| of type \verb|state|. We additionally define an abstract notion of state equality and a commutative state union operator. 
\begin{coq}
	state : Type;
	about_state : StrictlyComparable state;
	state0 : state;
	state_eq : state -> state -> Prop;
	state_union : state -> state -> state;
	state_union_comm : forall s1 s2, state_eq (state_union s1 s2) 
																						(state_union s2 s1);
\end{coq}
Further, we define a reachability relation on abstract states. We require of this reachability relation that it is reflexive and transitive. We relate reachability with the state union operator with the requirement that any state is able to reach its union with any other state. Finally, we require that reachability respects state equality. 
\begin{coq}
	reach : state -> state -> Prop;
	reach_refl : forall s, reach s s; 
	reach_trans : forall s1 s2 s3, reach s1 s2 -> reach s2 s3 -> reach s1 s3; 
	reach_union : forall s1 s2, reach s1 (state_union s1 s2);  
	reach_morphism : forall s1 s2 s3, reach s1 s2 -> state_eq s2 s3 -> reach s1 s3;  
\end{coq}

\subsubsection{Estimator} 
The estimator function is a total function which takes a state and returns a non-empty set of consensus values. We represent the estimator function using a relation of type \verb|state -> consensus_values -> Prop|, which can be interpreted as an assertion that the estimator agrees on some consensus value for some state. 
\begin{coq}
	E : state -> consensus_values -> Prop; 
	estimator_total : forall s, exists c, E s c; 
\end{coq}

\subsubsection{Protocol states}
We represent restricting elements of generic type \verb|state| to valid protocol states by defining protocol state as a property of type \verb|state -> Prop|. In this way, we abstract the mutually recursive definition of protocol states as sets of messages and messages as triples containing protocol states away into the implementation layer, and exclude the definition of messages from our type class. We require of the property \verb|prot_state| that it holds for the initial state. 
\begin{coq}
	prot_state : state -> Prop; 
	about_state0 : prot_state state0; 
\end{coq}

\subsubsection{Equivocation} 
Finally, we abstract away the details of computing fault weight from states by representing the fault weight of a state as a function of type \verb|state -> R|. We require that the fault weight of the union of two states is greater than or equal to the fault weight of each individual state, and we relate the definition of protocol states with fault weight by requiring that two protocol states' union is a protocol state iff its fault weight is under threshold. 
\begin{coq}
	equivocation_weight : state -> R; 
	equivocation_weight_compat : forall s1 s2, 
	(equivocation_weight s1 <= equivocation_weight (state_union s2 s1))%R; 
	about_prot_state : forall s1 s2, prot_state s1 -> prot_state s2 ->
										(equivocation_weight (state_union s1 s2) <= proj1_sig t)%R 
										-> prot_state (state_union s1 s2);
\end{coq}

\subsection{Safety} 
Next, we prove the asynchronous, Byzantine-fault-tolerant consensus safety properties of the family of CBC Casper protocols. Each proof assumes that we have a generic family member, i.e. an instance of the type class \verb|CBC_Casper_eq|. 
\subsubsection{Common futures} 
We prove that protocol states have common futures as long as their union state is a valid protocol state. We proceed stepwise, as in \cite{CBCfull}: first we show that any pair of protocol states satisfies the common futures property, then we use it to generalize to an arbitrary number of states. 
\begin{coq}
Theorem pair_common_futures '{CBC_protocol_eq}:
	forall s1 s2 : pstate,
	(equivocation_weight (state_union s1 s2) <= proj1_sig t)%R ->
	exists s : pstate, pstate_rel s1 s /\ pstate_rel s2 s.
	
Theorem n_common_futures "{CBC_protocol_eq} :
forall ls : list pstate,
(equivocation_weight (fold_right state_union state0 (map (fun ps => proj1_sig ps) ls)) <= proj1_sig t)%R ->
exists ps : pstate, Forall (fun ps' => pstate_rel ps' ps) ls.
\end{coq}	

\subsubsection{Consistency of decisions on protocol states}
We prove that protocol states which have common futures cannot make conflicting decisions on future states. We again proceed stepwise, first showing that any two protocol states with a common future cannot make conflicting decisions on future states. 
\begin{coq}
Theorem pair_consistency_prot '{CBC_protocol_eq} :
	forall s1 s2 : pstate,
	(equivocation_weight (state_union s1 s2) <= proj1_sig t)%R ->
	forall P, 
	~ (decided P s1 /\ decided (not P) s2).
\end{coq}
We then generalize this result to an arbitrary number of states. We generalize the definition of consistent decisions to a list of states. 
\begin{coq}
Definition state_consistency '{CBC_protocol_eq} (ls : list pstate) : Prop :=
exists s : pstate,
forall (P : pstate -> Prop),
Exists (fun s => decided P s) ls ->
P s.
\end{coq}
We then state n-party consensus value consistency as follows: 
\begin{coq}
Theorem n_consistency_prot '{CBC_protocol_eq} :
forall ls : list pstate,
(equivocation_weight (fold_right state_union state0 (map (fun ps => proj1_sig ps) ls)) <= proj1_sig t)%R ->
state_consistency ls.
\end{coq}

\subsubsection{Consistency of decisions on consensus values} 
Showing that protocol states which have common futures cannot make conflicting decisions on consensus values follows naturally from the fact that consensus values are parameterized by state. The statement of consensus value safety is as follows: 
\begin{coq}
Theorem n_consistency_consensus '{CBC_protocol_eq} :
	forall ls : list pstate,
	(equivocation_weight (fold_right state_union state0 (map (fun ps => proj1_sig ps) ls)) <= proj1_sig t)%R ->
	consensus_value_consistency ls. 
\end{coq}

\section{Full node} 
\label{sec:full}
\subsection{States, messages, representatives}

The full-node protocol states are {\em sets} of messages,
each message being a triple $(c, v, j)$, where:
\begin{itemize}
    \item $c$ is a (proposed) consensus value
    \item $v$ identifies the sender of the message
    \item $j$, the justification, {\em is the protocol state} in which the sender
        was at the time the message was sent
\end{itemize}

There are two technical difficulties with the above definition:  
\begin{itemize}
    \item It is recursive (states are sets of messages, each containing a state)
    \item The ordering of the message should not matter
\end{itemize}

To solve the first issue, we chose to postpone the definition of messages and
to define states inductively as follows:

\begin{coq}
Inductive state : Type :=
  | Empty : state
  | Next : C ->  V -> state -> state -> state.
\end{coq}

This definition says that a state can be build by extending an existing state
given a consensus value (C), a validator (V), and another state representing
the justification.
To clarify that the three are the components of a message, we introduce the
following notation for \verb"Next":

\begin{coq}
Notation "'add' ( c , v , j ) 'to' sigma" :=
  (Next c v j sigma)
  (at level 20).
\end{coq}

This definitions addresses the first issue mentioned above, that is, the 
recursive nature of states.

The second issue is very important as it relates to the notion of
state equality.
Although it is pottentially possible to define an equivalence between
states which disregards the order between messages, that is definitely
not trivial, because it is itself recursive, as it requires an equivalence
on messages, which must be defined in terms of the same state equivalence.
This would be a hinderance to both define it and work with it.

Our approach was to work with canonical representatives of these classes,
namely what we called LocallySorted states.  Although this definition
is still recursive, it's easier to express and to work with it than with the
equivalence one because it is only defined in terms of a single state.

To define the notion of sorted states, we need to be able to compare messages,
which amounts to defining a (total) order relation on states.
This can be defined as a lexicographic ordering on the state seen as a list
of messages, with the tweak of recursing in order to compare justifications:

\begin{coq}
Fixpoint state_compare (sigma1 sigma2 : state) : comparison :=
  match sigma1, sigma2 with
  | Empty, Empty => Eq
  | Empty, _ => Lt
  | _, Empty => Gt
  | add (c1, v1, j1) to sigma1, add (c2, v2, j2) to sigma2 =>
    match compare c1 c2 with
    | Eq =>
      match compare v1 v2 with
      | Eq =>
        match state_compare j1 j2 with
        | Eq => state_compare sigma1 sigma2
        | cmp_j => cmp_j
        end
      | cmp_v => cmp_v
      end
    | cmp_c => cmp_c
    end
  end.
\end{coq}

Note that defining this ordering requires a existing orderings on
consensus values and validators, but these can any orderings,
and an total ordering is guaranteed to exists for any set in set theory
including axiom of choice\footnote{Dense orderings, partitions and weak
forms of choice, by Carlos G. Gonzalez FUNDAMENTA MATHEMATICAE 147 (1995)}.
    
This ordering naturally induces an ordering on messages, and thus allows us
to define the notion of a \verb"LocallySorted" state, i.e., one in which
each message is smaller than the next and all justifications are themselves
\verb"LocallySorted".

\verb"LocallySorted" states are chosen as representatives for states, with the
benefit that testing equality of such states reduces to checking syntactic
equality.

Protocol states are defined by means of an inductive predicate on states, by

\begin{coq}
(* Valid protocol state definition *) 
Inductive protocol_state : state -> Prop :=
  | protocol_state_empty : protocol_state Empty
  | protocol_state_next
         : forall s j
         , protocol_state s
        -> protocol_state j
        -> incl_messages j s
        -> forall c v
         , valid_estimate c j
        -> not_heavy (add_in_sorted_fn (c,v,j) s)
        -> protocol_state (add_in_sorted_fn (c,v,j) s).
\end{coq}

The above definition reads as:
\begin{itemize}
    \item a protocol state is either empty; or
    \item it can be obtained from an existing protocol state $s$ by extending
        it with a message $(c, v, j)$ such that:
        \begin{itemize}
            \item $j$ is a protocol state included in $s$
            \item $c$ is a consensus values which can be estimated by $j$
            \item adding $(c,v,j)$ to $s$ does not produce a heavy state
        \end{itemize}
\end{itemize}

\subsection{Strong-non-triviality proofs}
Emphasis on explicating: 
\begin{enumerate}
	\item strong non-triviality proof: atomic equivocation construction, pivotal validator proof, recursive atomic equivocation construction, overall proof sketch
\end{enumerate}




\section{Light node}
\label{sec:light}
\subsection{States, hashes, messages, representatives}

The light-node protocol states are also sets of messages,
each message being a triple $(c, v, j)$, where:
\begin{itemize}
    \item $c$ is a (proposed) consensus value
    \item $v$ identifies the sender of the message
    \item $j$, the justification, is the {\em set of hashes} of all messages
        of the protocol state in which the sender was at the time the message
        was sent
\end{itemize}

The fact that justification is a set of hashes allows for a simpler,
non-recursive definition. To begin, the type of hashes can be any
totally ordered set, a justification can be a list of hashes, and
a message is now just a triple:

\begin{coq}
Variable hash : Type.
Variable (about_H : `{StrictlyComparable hash}). 

Definition justification_type : Type := list hash. 

Definition message : Type := C * V * justification_type.
\end{coq}

The total order on hashes induces a total lexicographic order on
justifications, which can be extended to messages.

We therefore can work with sorted lists of hashes as representatives
for sets of hashes, reducing equality between justifications to
syntactic equality.

For states we prefer to define equality as set-equality, that is
double inclusion between the sets of messages representing states.

Assuming an injective function from messages to hashes,
\begin{coq}
Parameters (hash_message : message -> hash)
           (hash_message_injective : Injective hash_message).
\end{coq}    
we can recursively define a function \verb"hash_state" taking 
states to sorted lists of hashes, i.e., justifications, with the
property that two justifications are equal iff they belong to 
states which are equal as sets:

\begin{coq}
Lemma hash_state_injective : forall sigma1 sigma2,
  hash_state sigma1 = hash_state sigma2
  <->
  set_eq sigma1 sigma2.
\end{coq}    

This allows for the following inductive definition of protocol states:

\begin{coq}
Inductive protocol_state : state -> Prop :=
| protocol_state_nil : protocol_state state0
| protocol_state_cons : forall (j : state),
    protocol_state j ->
    forall (c : C),
      valid_estimate c j ->
      forall (v : V) (s : state),
        In (c, v, hash_state j) s ->
        protocol_state (set_remove compare_eq_dec (c, v, hash_state j) s) ->
        NoDup s ->
        not_heavy s ->
        protocol_state s.
\end{coq}

The above definition reads as:
\begin{itemize}
    \item a protocol state is either empty; or
    \item it is a non-heavy, non-duplicate state $s$ for which
        there exist a consensus value $c$, a sender $V$, and a state $j$ such that:
        \begin{itemize}
            \item $j$ is a protocol state
            \item $c$ is a consensus values which can be estimated by $j$
            \item $(c,v, \texttt{hash\_state}\  j)$ belongs to $s$
            \item The state obtained from $s$ by removing
                $(c,v, \texttt{hash\_state}\  j)$ is a protocol state.
        \end{itemize}
\end{itemize}



\section{Related work and discussion}
\label{sec:related}
In terms of related work, of significance is the recent formalization of Casper CBC in Isabelle \cite{Nakamura}.
Our approach resembles that of Nakamura et al. from a high-level point of view, however, we try to keep the protocol as abstract as possible and formalize the results as closely as possible to their statement in the Casper CBC white papers.
Further, while \cite{Nakamura} only tackles the full node version of Casper TFG and formulate and prove for it a slightly different version of safety, we prove safety, non-triviality, and strong non-triviality for both the full node and light node abstractions of Casper CBC.
On a more technical note, their definition of full node messages uses lists instead of sets of messages for justification, ``because sets can not be used for recursive definition'' \cite{Nakamura-github}; to capture the intended idea of sets we use a more involved definition for states and ensure that protocol states are always sorted.

Our future directions for CBC Casper in Coq include further developing our abstraction hierarchy to include strong non-triviality. We are also interested to explore whether other consensus protocols can fit at various levels of this hierarchy. Additionally, we intend to develop and prove further properties about CBC Casper, including liveness. 


\section{Conclusion}
\label{sec:conclusion}


% Bibliography has been alphabetically rearranged according to last name of first author.
\begin{thebibliography}{99}

\bibitem{CiC} Thierry Coquand, Gérard Huet.
\newblock The Calculus of Constructions.
\newblock {\em Information and Computation}, 76(2-3):95--120, 1988.

\bibitem{Coq} 
\newblock The Coq Proof Assistant. 
\newblock https://coq.inria.fr/.

\bibitem{Nakamura} Ryuya Nakamura, Takayuki Jimba, Dominik Harz.
\newblock Refinement and Verification of CBC Casper. 
\newblock {\em Cryptology ePrint Archive}, Report 2019/415, 2019.

\bibitem{CBCfull} Vlad Zamfir, Nate Rush, Aditya Asgaonkar, Georgios Piliouras.
\newblock Introducing the ``Minimal CBC Casper'' Family of Consensus Protocols. 
\newblock 2019. 

\bibitem{CBClight} Vlad Zamfir
\newblock Introducing the ``Minimal'' CBC Casper Light Client -- with some explorations. 
\newblock 2019. 
\end{thebibliography}


%%\begin{thebibliography}{00}

%% \bibitem{label}
%% Text of bibliographic item

%%\end{thebibliography}
\end{document}
\endinput
%%
%% End of file `elsarticle-template-num.tex'.
