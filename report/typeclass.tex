The properties desired of CBC Casper broadly fall into two categories: 1) safety and 2) non-triviality. Definition of what these two things are. Some intuition on why these two things are important. We show below our incremental approach to proving these properties, starting from a definition of CBC Casper in terms of a state transition system with a reflexive and transitive reachability relation, also known in rewrite logic literature as a partial order and in modal logic literature as a KT4 Kripke model.
\subsection{Partial order} 
\begin{lstlisting}
Class PartialOrder :=
{ A : Type;
	A_eq_dec : forall (a1 a2 : A), {a1 = a2} + {a1 <> a2};
	A_inhabited : exists (a0 : A), True; 
	A_rel : A -> A -> Prop;
	A_rel_refl :> Reflexive A_rel;
	A_rel_trans :> Transitive A_rel;
}.
\end{lstlisting}
At this level, we are able to derive all of the safety properties desired of CBC Casper, namely: 
\begin{lstlisting}
Theorem pair_common_futures '{CBC_protocol_eq}:
forall s1 s2 : pstate,
(equivocation_weight (state_union s1 s2) <= proj1_sig t)%R ->
exists s : pstate, pstate_rel s1 s /\ pstate_rel s2 s.

Theorem n_common_futures '{CBC_protocol_eq} :
forall ls : list pstate,
(equivocation_weight (fold_right state_union state0 (map (fun ps => proj1_sig ps) ls)) <= proj1_sig t)%R ->
exists ps : pstate, Forall (fun ps' => pstate_rel ps' ps) ls.

Theorem pair_consistency_prot '{CBC_protocol_eq} :
forall s1 s2 : pstate,
(equivocation_weight (state_union s1 s2) <= proj1_sig t)%R ->
forall P, 
~ (decided P s1 /\ decided (not P) s2).

Theorem n_consistency_prot '{CBC_protocol_eq} :
forall ls : list pstate,
(equivocation_weight (fold_right state_union state0 (map (fun ps => proj1_sig ps) ls)) <= proj1_sig t)%R ->
state_consistency ls.

Theorem n_consistency_consensus '{CBC_protocol_eq} :
forall ls : list pstate,
(equivocation_weight (fold_right state_union state0 (map (fun ps => proj1_sig ps) ls)) <= proj1_sig t)%R ->
consensus_value_consistency ls. 
\end{lstlisting}
These results correspond to Theorems X - Y in \cite{CBCfull}. 
\newline


\subsection{Partial order with non-local confluence} 
In order to prove non-triviality properties, we additionally require  that our partial order possess certain confluence properties. In fact, we find that non-triviality as defined in \cite{CBCfull} directly captures the notion of non-local confluence in state transition systems, defined abstractly as follows: 

\begin{lstlisting}
Class PartialOrderNonLCish '{PartialOrder} :=
{ no_local_confluence_ish : exists (a a1 a2 : A),
														A_rel a a1 /\ A_rel a a2 /\
														~ exists (a' : A), A_rel a1 a' /\ A_rel a2 a';
}. 
\end{lstlisting}

\subsection{Abstract protocol} 
To provide a richer, protocol-specific language to describe our desired properties, we give an abstract type class from which we can generalize a partial order, but which contains information specific to consensus protocols, including types for validators, consensus values, states, and an abstract, total estimator function. 
\begin{lstlisting}
Class CBC_protocol_eq :=
{
consensus_values : Type; 
about_consensus_values : StrictlyComparable consensus_values; 
validators : Type; 
about_validators : StrictlyComparable validators;
weight : validators -> {r | (r > 0)%R}; 
t : {r | (r >= 0)%R}; 
suff_val : exists vs, NoDup vs /\ ((fold_right (fun v r => (proj1_sig (weight v) + r)%R) 0%R) vs > (proj1_sig t))%R;
state : Type;
about_state : StrictlyComparable state;
state0 : state;
state_eq : state -> state -> Prop;
state_union : state -> state -> state;
state_union_comm : forall s1 s2, state_eq (state_union s1 s2) (state_union s2 s1);
reach : state -> state -> Prop;
reach_refl : forall s, reach s s; 
reach_trans : forall s1 s2 s3, reach s1 s2 -> reach s2 s3 -> reach s1 s3; 
reach_union : forall s1 s2, reach s1 (state_union s1 s2);  
reach_morphism : forall s1 s2 s3, reach s1 s2 -> state_eq s2 s3 -> reach s1 s3;  
E : state -> consensus_values -> Prop; 
estimator_total : forall s, exists c, E s c; 
prot_state : state -> Prop; 
about_state0 : prot_state state0; 
equivocation_weight : state -> R; 
equivocation_weight_compat : forall s1 s2, (equivocation_weight s1 <= equivocation_weight (state_union s2 s1))%R; 
about_prot_state : forall s1 s2, prot_state s1 -> prot_state s2 ->
(equivocation_weight (state_union s1 s2) <= proj1_sig t)%R -> prot_state (state_union s1 s2);
}.
\end{lstlisting}

The plan here is to 1) say that our t

We prove that \verb|CBC_protocol_eq| can derive \verb|PartialOrder|. 